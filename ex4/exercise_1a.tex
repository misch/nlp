\documentclass[a4paper]{article}

\usepackage[english]{babel}
\usepackage[utf8]{inputenc}
\usepackage{amsmath}
\usepackage{graphicx}
\usepackage[colorinlistoftodos]{todonotes}

\begin{document}
\section*{Exercise 1.a)}
We are given a text of $n = 1'000'000$ words. The word \textit{Agatha} occurs $30$ times, \textit{Christie} has $117$ occurrences and the bigram \textit{Agatha Christie} occurs $20$ times.

\vspace{5mm}
\noindent From the given data we can directly compute the probabilities
\begin{align*}
P(\textit{Agatha}) &= \frac{30}{1'000'000} = 3 \cdot 10^{-5}, \\
P(\textit{Christie}) &= \frac{117}{1'000'000} = 0.000117. \\
\end{align*}
Following the Null-Hypothesis ($H_0$) of \textit{Agatha} and \textit{Christie} being independent, we can estimate the probability of the bigram \textit{Agatha Christie} as 

$$p_0 = P(\textit{Agatha}) \cdot P(\textit{Christie}) = 3.51 \cdot 10^{-9}.$$

\noindent However, directly computing the probability of the bigram from the given data yields

$$p = P(\textit{Agatha Christie}) = \frac{20}{1'000'0000} = 2 \cdot 10^{-5}.$$

\noindent If we see this as a Bernoulli process, the mean and variance are given by

\begin{align*}
\bar{X} &= p = 2 \cdot 10^{-5}, \\
s^2 &= p \cdot (1-p) \approx 1.2 \cdot 10^{-5}.
\end{align*}

\noindent The mean value from the Null-Hypothesis is given by 
$$ \mu_0 = p_0 = 3.51 \cdot 10^{-9}$$

\noindent Therefore, we get a t-value of 
$$t_{obs} = \frac{\bar{X} - \mu_0}{\sqrt{\frac{s^2}{n}}} \approx 4.471 $$


\noindent From the normal table, with a significance level of $1\%$ and dof\footnote{dof = degrees of freedom} = $\infty$, we can get the critical value $t_{lim} = 2.576$. The fact that $t_{obs} > t_{lim}$ means that the Null-Hypothesis is rejected. Therefore, the words \textit{Agatha} and \textit{Christie} are \textbf{not} independent.

\end{document}
