\documentclass[a4paper]{article}

\usepackage[english]{babel}
\usepackage[utf8]{inputenc}
\usepackage{amsmath}
\usepackage{graphicx}
\usepackage[colorinlistoftodos]{todonotes}

\begin{document}
\section*{Exercise 1.b)}
Following the exact same steps as in exercise 1. a), we get the following values:

\begin{align*}
P(\textit{Agatha}) &= \frac{200}{1'000'000} = 0.0002, \\
P(\textit{Christie}) &= \frac{300}{1'000'000} = 0.0003. \\
\end{align*}
With the Null-Hypothesis we get the mean value

$$\mu_0 = p_0 = P(\textit{Agatha}) \cdot P(\textit{Christie}) = 6 \cdot 10^{-8}.$$

\noindent Directly computation from the data gives the sample mean and sample variance

\begin{align*}
\bar{X} &= p = P(\textit{Agatha Christie}) = \frac{6}{1'000'000} = 6 \cdot 10^{-6}, \\
s^2 &= p \cdot (1-p) \approx p = 6 \cdot 10^{-6}.
\end{align*}

\noindent Therefore, we get a t-value of 
$$t_{obs} = \frac{\bar{X} - \mu_0}{\sqrt{\frac{s^2}{n}}} \approx 2.425 $$


\noindent From the normal table, with a significance level of $1\%$ and dof\footnote{dof = degrees of freedom} = $\infty$, we can get the critical value $t_{lim} = 2.576$. The fact that $t_{obs} < t_{lim}$ means that the Null-Hypothesis is not rejected. Therefore, the words \textit{Agatha} and \textit{Christie} appear independently in the text.

\end{document}