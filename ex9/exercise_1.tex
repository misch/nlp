\documentclass[a4paper]{article}

\usepackage[english]{babel}
%\usepackage{geometry}
%\geometry{a4paper,left=25mm,right=30mm} 
\usepackage{geometry}
\usepackage[utf8]{inputenc}
\usepackage{amsmath}
\usepackage{graphicx}
\usepackage{slashbox}
%\usepackage[colorinlistoftodos]{tod4notes}
\usepackage{qtree}
%\usetikzlibrary{arrows}
\newcommand{\lra}{$\longrightarrow$ }
\newcommand{\f}{-- \textbf{fail}}
\newcommand{\fact}{\text{Factor }}
\newcommand{\term}{\text{Term }}
\newcommand{\idf}{\text{idf}}
\newcommand{\df}{\text{df}}
\newcommand{\tf}{\text{tf}}
\newcommand{\ti}{\text{tf-idf}}
\newcommand{\cossim}{\text{cosSim}}

\begin{document}
\section*{Exercise 1}
\subsection*{a) Compute idf}
The value of the idf (inverse document frequency) for a particular term $T$ is given by
$$\idf_T = \log\left(\frac{n}{\df_T}\right),$$
where $n$ is the \emph{total number of documents in the corpus}, in our example $n = 10'000$, and $\df_T$ denotes the \emph{document frequency} of term $T$, i.e. the number of documents that contain $T$.
From our example we have the following document frequencies:

\begin{align*}
 \df_{T_1} &= 100, \\
 \df_{T_2} &= 200, \\
 \df_{T_3} &= 200, \\
 \df_{T_4} &= 100.
\end{align*}

From this we can directly compute the idf-values for all our 4 terms:
\begin{align*}
 \idf_{T_1} &= \idf_{T_4} = \log_{10} \left(\frac{10'000}{100} \right) = 2, \\
 \idf_{T_2} &= \idf_{T_3} = \log_{10} \left(\frac{10'000}{200} \right) \approx 1.69897
\end{align*}

\subsection*{b) Compute tf-idf weighting}
The tf-idf for a term $T_k$ and a document $d_j$ is given by 
$$\ti_{kj} = \tf_{kj} \cdot \idf_{T_k}$$
where $t_{kj}$ denotes the number of occurrences of term $T_k$ in document $d_j$, i.e. the numbers that are given in the table.
With the above formula we get the following tf-idf weights:

\begin{table}[h]
\center
\begin{tabular}{| c | c | c | c | c |}
\hline
\textbf{Document \textbackslash Term} & $T_1$ & $T_2$ & $T_3$ & $T_4$ \\ \hline
$D_1$ & $4 \cdot 2 = 8$ & $4 \cdot 1.69897 = 6.79588$ &  $0$  & $ 1\cdot 2 = 2$ \\
$D_2$ & $8$ & $2 \cdot 1.69897 = 3.39794$ & $10 \cdot 1.69897 = 16.9897$ & $ 5 \cdot 2 = 10$ \\ 
$D_3$ & $8$ & $3.39794$ & $2 \cdot 1.69897 = 3.39794$ & $30 \cdot 2 = 60$ \\
\hline
\end{tabular}
\label{tab:tf-idf}
\caption{tf-idf values }
\end{table}
\subsection*{c) Order of documents for query $(T_3,T_4)$}
Assume we have a document $D_4$ consisting only of the two terms $T_3$ and $T_4$. We can compute the tf-idf value for this document as follows: 

\begin{table}[h]
\begin{tabular}{| c | c | c |}
\hline
Term & $T_3$ & $T_4$ \\ \hline
tf-idf & $1 \cdot 1.69897 = 1.69897$ & $1 \cdot  2 = 2$\\ \hline 
\end{tabular}
\end{table}

\vspace{5mm}
\noindent The terms $T_1$ and $T_2$ will of course have a tf-idf of $0$ because neither of them occur in the given document. Representing each document as a vector of the above computed tf-idf weights gives the following vectors:
\begin{align*}
 D_1 &= (8, 6.79588, 0, 2), \\
 D_2 &= (8, 3.39794, 16.9897, 10 ), \\
 D_3 &= (8, 3.3979,3.39794, 60), \\
 D_4 &= (0,0,1.69897, 2)
\end{align*}

\noindent We can compare two documents by comparing their relativ angle -- or rather by looking at the cosine between two vector, which can be found by building the dot product between two normalized vectors:
\begin{align*}
 &(v_1,v_2) = \cos(\theta) \cdot |v_1| \cdot |v_2| \\
 &\Longleftrightarrow \cos(\theta) = \frac{(v_1,v_2)}{|v_1| \cdot |v_2|},
\end{align*}
where $(v_1,v_2)$ denotes the dot product between two vectors $v_1$ and $v_2$, i.e. 
$$\sum_i v_1^{(i)}\cdot v_2^{(i)},$$
and $|v_i]$ is given by $\sqrt{(v_i,v_i)}$.

\vspace{5mm}
\noindent Using this comparison scheme, we can compute scores for the comparison of $D_4$ with all the other documents. We start by computing the lengths of the single vectors:
\begin{align*}
 |D_1| &= \sqrt{8^2 + 6.79588^2 + 0^2 + 2^2} \approx 10.6857 \\
 |D_2| &\approx 21.5452 \\
 |D_3| &\approx 60.7214 \\
 |D_4| &\approx 2.62421 \\
\end{align*}
The cosine similarities are now given by
\begin{align*}
 \cossim(D_1,D_4) &= \frac{(D_1,D_4)}{|D_1|\cdot |D_4|} = \frac{8 \cdot 0 + 6.79588 \cdot 0 + 0  \cdot 1.69897 + 2^2}{10.6857 \cdot 2.62421} = \frac{2^2}{10.6857 \cdot 2.62421}  \approx 0.1426 \\
 \cossim(D_2,D_4) &= \frac{16.9897 \cdot 1.69897 + 10 \cdot 2}{21.5452 \cdot 2.62421} \approx 0.8643 \\
 \cossim(D_3,D_4) &= \frac{3.39794 \cdot 1.69897 + 60 \cdot 2}{60.7214 \cdot 2.62421} \approx 0.7893
\end{align*}

\noindent Therefore we have the ordering $D_2 > D_3 > D_1$, which means that document $D_4$ is most similar to $D_2$.
\end{document}