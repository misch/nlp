\documentclass[a4paper]{article}

\usepackage[english]{babel}
\usepackage{geometry}
\geometry{a4paper,left=25mm,right=30mm} 
\usepackage{geometry}
\usepackage[utf8]{inputenc}
\usepackage{amsmath}
\usepackage{graphicx}
\usepackage[colorinlistoftodos]{todonotes}
\usepackage{qtree}
\usetikzlibrary{arrows}
\newcommand{\lra}{$\longrightarrow$ }
\newcommand{\f}{-- \textbf{fail}}
\newcommand{\fact}{\text{Factor }}
\newcommand{\term}{\text{Term }}
\newcommand{\oor}{ ~ \vert ~ }


\title{Your Paper}

\author{You}

\date{\today}

\begin{document}
\section*{Exercise 4}
$G = (N,T,P,S)$. The terminals $T$ are given by $T = \{\text{number}, \text{variable}, +, -, \cdot, /\}$, the non-terminals are given by $N = \{ \text{Term}, \text{Factor}, S, A \}$. we have the following production rules $P$:

\begin{align*}
 S &\longrightarrow \term \oor \term \text{addExpression } S \\
 \text{Term} &\longrightarrow \fact \oor \fact \text{multExpression } \fact \\
 \fact &\longrightarrow \text{number} \oor \text{variable} \\
 \\
 \text{addExpression} &\longrightarrow + \oor - \\
 \text{multExpression} &\longrightarrow \cdot \oor /
\end{align*}

\noindent The fact that the start symbol $S$ only allows Terms but not Factors as following non-terminals, ensures that the multiplication/division are parsed as nodes \emph{within} terms. 
One could also understand it as first splitting the nodes at the +'s in order to not wrongly split factors.

\subsection*{example}
\begin{itemize}
 \item The only matching parse tree for $1 + 2\cdot a + b$ is: \\ \\ \Tree [.S  [.Term [.Factor [number ]] ] [.addExpression + ] !\qsetw{0.1cm} [.S [.Term [.Factor number ] [.multExpression $\cdot$ ] [.Factor variable ]] [.addExpression + ] !\qsetw{0.1cm} [.S [.Term [.Factor variable ]]]]]
\end{itemize}


\end{document}